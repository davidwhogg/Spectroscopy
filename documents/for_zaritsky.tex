\documentclass[12pt, letterpaper]{article}

\newcommand{\project}[1]{\textsl{#1}}
\newcommand{\SDSS}{\project{SDSS}}

\begin{document}

\section*{Cross-correlations between nearby galaxies and H$\alpha$ photons}

\noindent
\textsl{by DWH for DZ}\\
draft 2014-07-20

\bigskip

During a seminar by Kollmeier (OCIW) at NYU about the metagalactic
(should be ``intergalactic'') radiation field, DZ suggested that we
look for H$\alpha$ photons from the outskirts of low-redshift galaxies
using the spectroscopic footprint of the \project{Sloan Digital Sky
  Survey}.
This document represents an attempt to make this project well-posed.
The basic idea is to perform something akin to a galaxy--photon
cross-correlation using lines of sight that are background to (higher
redshift than) the low-redshift galaxy targets.

What we want to compute (perhaps) is the mean intensity (energy per
solid angle per area per time per wavelength) in the
wavelength-vicinity of the H$\alpha$ line as a function of projected
radius away from the center of each of or a set of low-redshift
galaxies.

What we need as inputs are the following:
\begin{itemize}
\item
A list of nearby target galaxies $n$, each of which has a sky position
$\theta_n$, a redshift $z_n$, and some kind of angular size $\phi_n$.
These galaxies will be chosen to have redshifts such that $z_1 < z_n <
z_2$, where $z_1$ and $z_2$ are some limits.  We will presume that the
positions, redshifts, and angular sizes are well known.
\item
A list of background sources $m$, each of which has a sky position
$\theta_m$, a redshift $z_m$, and an \SDSS\ spectrum $f_m$.  These
background sources will be chosen to have redshifts $z_m > z_3$, where
$z_3 > z_2$.  For each of these background sources, we will presume
that there is not only an \SDSS\ spectrum $f_m$ but also some kind of
model prediction $\mu_m$ (for the flux as a function of wavelength) so
that we can get \SDSS\ data-minus-model residuals (we will need these
soon).  Included in this list of background sources is all the
\SDSS\ sky spectra, which have (effectively) infinite redshift and a
model prediction of vanishing flux.
\item
A random catalog of ersatz target galaxies, each of which has a
redshift and size drawn from the target galaxy list, but a random sky
position drawn from the \SDSS\ footprint.
\item
A random catalog of ersatz background soures, each of which has a
redshift and spectrum drawn from the background source list, but a
random sky position drawn from the \SDSS\ footprint.
\end{itemize}

...Fundamentally, our estimator will be something like $DD - DR$ or else
$DD - DR - RD + RR$...

...In detail, these terms are...

...Each line of sight is a noisy measurement of the intensity field, so
we can do inverse-variance-weighted averaging...

[yada yada]

Unresolved or open issues include the following:
\begin{itemize}
\item
Should we be doing work on the spectral pixels, or should we be doing
work on some kind of line-intensity measurement on each spectrum, at
the relevant H$\alpha$ wavelength?
\item
Should we do the statistical cross-correlation first, or should we try
things out on a single object (low-redshift Milky-Way-like galaxy)
first?  There is nothing about our plan that requires a large sample
of targets.
\item
Should we adjust the method for galactic rotation curves, or can we
assign one redshift to every galaxy?  Should we do anything at the
spectrum level to deal with velocity gradients?
\item
What should be $z_1$, $z_2$, and $z_3$?  In principle, $z_1$ should be
large enough that we don't get messed up by telluric H$\alpha$.  In
principle, $z_3$ should be larger than $z_2$ by enough that we don't
get overlapping H$\alpha$ lines.
\end{itemize}

\end{document}
