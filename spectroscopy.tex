\documentclass[12pt]{article}
\usepackage{amssymb,amsmath,color,hyperref}
\definecolor{linkcolor}{rgb}{0,0,0.25}
\hypersetup{
  colorlinks=true,        % false: boxed links; true: colored links
  linkcolor=linkcolor,    % color of internal links
  citecolor=linkcolor,    % color of links to bibliography
  filecolor=linkcolor,    % color of file links
  urlcolor=linkcolor      % color of external links
}
\newcommand{\hmatrix}[1]{\boldsymbol{#1}}
\newcommand{\Amatrix}{\hmatrix{A}}
\newcommand{\pixels}{\hmatrix{p}}
\newcommand{\biaspixels}{\hmatrix{p}^\mathrm{(bias)}}
\newcommand{\darkpixels}{\hmatrix{p}^\mathrm{(dark)}}
\newcommand{\arcpixels}{\hmatrix{p}^\mathrm{(arc)}}
\newcommand{\flatpixels}{\hmatrix{p}^\mathrm{(flat)}}
\newcommand{\bias}{\hmatrix{b}}
\newcommand{\dark}{\hmatrix{d}}
\newcommand{\flux}{\hmatrix{f}}
\newcommand{\sky}{\hmatrix{s}}
\newcommand{\arc}{\hmatrix{a}}
\newcommand{\lamp}{\hmatrix{q}}
\renewcommand{\angle}{\hmatrix{\theta}}
\newcommand{\error}{\hmatrix{e}}
\newcommand{\exptime}{t}

\begin{document}

In a modern spectrograph, light from fibers or slits is dispersed onto
a two-dimensional CCD or CCD-like detector, with (usually) one
direction on the detector corresponding to an angular displacement on
the sky and another (usually) close-to-orthogonal direction
corresponding to wavelength.  There are also slitless spectrographs,
where one direction is a mixture of angular position and wavelength.
The problem of spectroscopic data reduction is the problem of
extracting the one-dimensional spectra---astronomical source flux
densities as a function of wavelength---from the two-dimensional
images.  There is a literature on ``optimal extraction'' of these
one-dimensional spectra; the best extraction methods treat the
two-dimensional image pixels as data to be fit by a model that
consists of a one-dimensional spectrum laid out geometrically on the
device and convolved with a two-dimensional point-spread function
(which might be a function of the atmospheric seeing or the device
properties or both).

Although the optimal extraction literature solves some important
problems, the hardest part of spectroscopic data reduction lies not in
the extraction step but in the step of measuring or learning the
geometric and point-spread functions themselves.  These functions have
various names in the literature but they can be expressed with a
single rectangular matrix $\Amatrix$:
\begin{equation}
\Delta\pixels = \exptime\,\Amatrix\,\flux \quad ,
\end{equation}
where the column vector $\pixels$ is a one-dimensional matrix listing
all of the pixel values in the two-dimensional detector in some order,
the column vector $\Delta\pixels$ lists the differential contribution
to those pixels from an individual astronomical object (that is, the
contribution over and above the bias, dark, sky, and any residual
contributions from other spatially separated sources), the scalar
$\exptime$ is the exposure time of the image, and the column vector
$\flux$ is a list of the flux densities on a grid or list of
wavelength values coming from that astronomical object.  In this
description, each element of the matrix $\Amatrix$ gives the rate of
change of a particular pixel in the two-dimensional detector in
response to unit flux density at a particular wavelength from the
source.  For simplicity, we will concentrate here on sources that are
compact or else fiber or slit spectra with small spectral apertures.

In what follows, we develop a method for inferring this matrix
$\Amatrix$, which contains almost all of the calibration information
necessary for the extraction of spectroscopic data.  Justifiable
objective methods for inferring $\Amatrix$ are not currently in
general use.  Most spectroscopic surveys use a combination of ad-hoc
methods to estimate wavelength solutions, geometric trace functions,
and point spread functions, and combine these to create approximations
to $\Amatrix$.  Our goal is to produce a flexible objective
methodology in which decisions are made by the data not by the
investigators, and that will be extensible to a wide range of
situations.  We will try to keep the discussion general, but where we
need to get specific we will work with the data from the SDSS-III BOSS
spectrographs.

Our generative model for the data---for a science exposure or data
taken of astronomical targets---is that the column vector $\pixels_i$
of pixel values in exposure $i$ can be expressed as a sum of the
following terms:
\begin{equation}
\pixels_i = \bias + \exptime_i\,\dark
          + \exptime_i\,\sum_j \Amatrix_j\,\flux_j
          + \exptime_i\,\sum_j \Amatrix_j\,\sky_i(\angle_j)
          + \error_i \quad ,
\end{equation}
where the column vector $\bias$ lists the bias or zero-level for every
pixel, the scalar $\exptime_i$ is the exposure time for exposure $i$,
the column vector $\dark$ lists the dark current rate for every pixel,
there is a rectangular matrix $\Amatrix_j$ for every spectroscopic
aperture or fiber $j$, the column vector $\flux_j$ lists the flux
density level for every wavelength in the wavelength grid for the
source illuminating aperture $j$, the column vector $\sky_i(\angle_j)$
lists the flux density (intensity integrated over the aperture) of the
sky, which is permitted to be different for different exposures and to
be a weak function of the angular position $\angle_j$ of aperture $j$,
and the column vector $\error_i$ lists the errors or residuals away
from the model in this exposure $i$.

There are also bias, dark, arc, and flat frames, which in our model
are generated by
\begin{eqnarray}\displaystyle
\biaspixels_i & = & \bias 
                + \error_i \quad ,\\
\darkpixels_i & = & \bias + \exptime_i\,\dark
                + \error_i \quad ,\\
\arcpixels_i  & = & \bias + \exptime_i\,\dark
                + \exptime_i\,\sum_j \Amatrix_j\,\arc(\angle_j)
                + \error_i \quad ,\\
\flatpixels_i & = & \bias + \exptime_i\,\dark
                + \exptime_i\,\sum_j \Amatrix_j\,\lamp(\angle_j)
                + \error_i \quad ,
\end{eqnarray}
where the error vector $\error_i$ will be different for every frame,
presumably drawn from a distribution function that depends on
illumination but is known at some level, the column vector
$\arc(\angle_j)$ lists the flux density values for an arc lamp,
probably modeled as a mixture of narrow emission lines at known
wavelengths, but permitted to vary in intensities slowly with the
angular position $\angle_j$ of aperture $j$, and the column vector
$\lamp(\angle_j)$ lists the flux density values for a flat-field lamp,
again with slow angular variations.

Our problem is to infer from these data---science exposures, biases,
darks, arcs, and flats---all of the matrices $\Amatrix_j$ for all of
the apertures $j$.

We will assume smoothness in various things.

We will specialize to BOSS spectra for now.

We will assume that each plate is not independent but has $\Amatrix$
drawn from some compact distribution.

We will assume that lamps and skies and arcs are not infinitely
variable from plate to plate and night to night.

\end{document}
